\documentclass{article}

\usepackage{preamble}

\title{Homework 5}
\author{Brandon Henke\\\textit{PHY841}\\\textit{Andreas von Manteuffel}}
\date{\today}

\setcounter{section}{5}

\begin{document}

\maketitle
\begin{multicols*}{2}

	\subsection{A Simple Poisson Equation}
	\subsubsection*{a}
	The Laplace operator, in cartesian coordinates, is given by
	\[
		\laplacian{} = \pdv[2]{x}+\pdv[2]{y}+\pdv[2]{z}.
	\]
	One can use a change of variables to find how the Laplace operator acts on a function that only depends on the distance from the origin, $f(r)$.
	The distance from the origin is $r = \sqrt{x^2+y^2+z^2}$.
	Taking the second derivative with respect to each of the cartesian coordinates is the same process:
	\begin{align*}
		\pdv[2]{x}
		&= \pdv{x}(\pdv{r}{x}\pdv{r}),\\
		&= \pdv[2]{r}{x}\pdv{r}+\left(\pdv{r}{x}\right)^2\pdv[2]{r},\\
		&= \left(\frac{1}{r}-\frac{x^2}{r^3}\right)\pdv{r} + \frac{x^2}{r^2}\pdv[2]{r},\\
		&= \frac{1}{r} \left(\left(1-\frac{x^2}{r^2}\right)\pdv{r}+\frac{x^2}{r}\pdv[2]{r}\right)
	\end{align*}
	Since the processes is the same for each variable,
	\begin{align*}
		\laplacian
		&= \frac{1}{r} \left(\left(3-\frac{r^2}{r^2}\right)\pdv{r}+\frac{r^2}{r}\pdv[2]{r}\right),\\
		&= \frac{1}{r} \left(2\pdv{r}+r\pdv[2]{r}\right),\\
		\laplacian f(r)
		&= \frac{1}{r}\pdv[2]{r}(rf(r)).
	\end{align*}

	\subsubsection*{b}
	The possion equation for this charge distribution is given by
	\begin{equation}
		\laplacian \phi(\vb{r}) = \frac{\rho_0}{\epsilon_0} e^{-\mu r}.
		\label{eq: poisson 1}
	\end{equation}
	Since the charge distribution is radially symmetric, the potential must also be radially symmetric, since rotating the system in any way doesn't change the problem.
	From part a, \ref{eq: poisson 1} becomes
	\[
		\frac{1}{r}\pdv[2]{r}(r\phi(r)) = \frac{\rho_0}{\epsilon_0} e^{-\mu r}.
	\]
	One can easily find the solution to this through integration\footnote{These integrals were done with the help of Mathematica.}:
	\begin{align*}
		\pdv{r}(r\phi(r)) &= \frac{\rho_0}{\epsilon_0} \int r e^{-\mu r}\dd{r},\\
		&= c_0-\frac{e^{-\mu r} (\mu r+1)}{\mu^2},\\
		r\phi(r) &= \int c_0-\frac{e^{-\mu r} (\mu r+1)}{\mu^2} \dd{r},\\
		&= c_1 + c_0 r + \frac{\rho_0 e^{-\mu r} \left(\frac{2}{\mu}+r\right)}{\mu^2 \epsilon_0},\\
		\phi(r) &= c_0 + \frac{c_1}{r} + \frac{\rho_0 e^{-\mu r} \left(\frac{2}{\mu}+r\right)}{\mu^2 \epsilon_0 r}.
	\end{align*}

	\subsubsection*{c}
	Since there is no charge density at infinity, the potential can be set to zero.
	Hence $c_0 = 0$V.
	Something weird is going on.
	Using the integral form of Gauss' law gives an electric field and potential that do not blow up at $r=0$m, yet solving Poisson's equation gave a solution that does blow up at $r=0$m.

	\subsection{Potential for a Box}
	Being that there is no charge in the box, Gauss' law becomes
	\begin{equation}
		\laplacian{\phi} = 0.
		\label{eq: laplace 2}
	\end{equation}
	The boundary conditions are
	\begin{align}
		\phi(0,y,z) = \phi(a,y,z) &= 0,
		\label{eq: boundary 2.1}\\
		\phi(x,0,z) = \phi(x,b,z) &= 0,
		\label{eq: boundary 2.2}\\
		\phi(x,y,0) &= 0,
		\label{eq: boundary 2.3}\\
		\phi(x,y,c) &= c_0 x,
		\label{eq: boundary 2.4}
	\end{align}
	where $c_0$ is some constant.

	\subsubsection*{a}
	Make the assumption that the solution to \ref{eq: laplace 2} is seperable:
	\[
		\phi(x,y,z) = X(x)Y(y)Z(z).
	\]
	Then \ref{eq: laplace 2} becomes
	\[
		\frac{X''}{X}+\frac{Y''}{Y}+\frac{Z''}{Z} = 0.
	\]
	For this to be the case, each term must be a constant, such that their sum vanishes:
	\begin{gather*}
		\frac{X''}{X} = -k_x^2,\\
		\frac{Y''}{Y} = -k_y^2,\\
		\frac{Z''}{Z} = k_z^2,\\
		k_z^2 = k_x^2 + k_y^2.
	\end{gather*}
	These ODEs have the following solutions:
	\begin{align*}
		X(x) &= X_0 \sin(k_x x) + X_1 \cos(k_x x),\\
		Y(y) &= Y_0 \sin(k_y y) + Y_1 \cos(k_y y),\\
		Z(z) &= Z_0 \sinh(k_z z) + Z_1 \cosh(k_z z).
	\end{align*}

	\subsubsection*{b}
	Using \ref{eq: boundary 2.1}-\ref{eq: boundary 2.3},
	\begin{gather*}
		X_1 = 0, \; k_x = \frac{m\pi}{a},\\
		Y_1 = 0, \; k_y = \frac{n\pi}{b},\\
		Z_1 = 0.
	\end{gather*}
	Thus,
	\begin{align*}
		X(x) &= X_0 \sin(\frac{m\pi}{a} x),\\
		Y(y) &= Y_0 \sin(\frac{n\pi}{b} y),\\
		Z(z) &= Z_0 \sinh(k_{mn}\pi z),
	\end{align*}
	where
	\[
		k_{mn} = \sqrt{\frac{m^2}{a^2}+\frac{n^2}{b^2}}.
	\]
	Using \ref{eq: boundary 2.4} gives
	\[
		V(x) = \sum_{m,n}^\infty A_{mn} \sin(\frac{m\pi}{a} x)\sin(\frac{n\pi}{b}y)
		\sinh(k_{mn}c\pi).
	\]
	In order to find the coefficients, $A_{mn}$, one can exploit the inner product
	\[
		\int_0^\tau \sin(\frac{m\pi}{\tau}s)\sin(\frac{n\pi}{\tau}s)\dd{s} = \frac{\tau}{2}\delta_{mn}.
	\]
	Multiplying both sides by
	\[
		\sin(\frac{m\pi}{a} x)\sin(\frac{n\pi}{b}y),
	\]
	and using the identity
	\[
		\int_0^b \sin[2](\frac{n\pi}{b}\eta)\dd{\eta} = \frac{b}{2},
	\]
	one can integrate over $x$ and $y$ to get\footnote{This integral was done with the assistance of Mathematica.}
	\begin{align*}
		A_{mn}
		&= \frac{b c_0}{2\sinh(k_{mn}c\pi)}\int_0^a \xi\sin[2](\frac{m\pi}{a}\xi)\dd{\xi},\\
		&= \frac{a^2 b c_0}{8\sinh(k_{mn}c\pi)}.
	\end{align*}
	This gives the final solution to be
	\[
		\phi(x,y,z) = \sum_{m,n}^\infty A_{mn} \sin(\frac{m\pi}{a} x)\sin(\frac{n\pi}{b}y)
		\sinh(k_{mn}z\pi),
	\]
	where
	\[
		A_{mn} = \frac{a^2 b c_0}{8\sinh(k_{mn}c\pi)} \qq{and} k_{mn} = \sqrt{\frac{m^2}{a^2}+\frac{n^2}{b^2}}.
	\]


\end{multicols*}
\end{document}
